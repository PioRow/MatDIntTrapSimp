\documentclass[a4paper,12pt]{article}
\usepackage{amsfonts}
\usepackage{amssymb}
\usepackage{latexsym}
\usepackage{amsmath}
\usepackage{amsthm}
\usepackage{graphicx}
\usepackage{polski}
\usepackage{indentfirst}
\usepackage[polish]{babel}
\usepackage{float}
\usepackage[T1]{fontenc}
\usepackage[utf8]{inputenc} % tu mo¿e byæ konieczne zast¹pienie "cp1250" przez np. "utf8"
\usepackage{setspace}
\usepackage{array}
\usepackage{multirow}
\usepackage{geometry}
\geometry{hdivide={2cm,*,2cm}}
\geometry{vdivide={2cm,*,2cm}}
\usepackage{titlesec}
\titlespacing{\section}{0ex}{1ex}{1ex} % zmniejszenie odstêpów przed i po tytule rozdzia³u...
\titleformat*{\section}{\sf\large\bfseries} % i zmiana kroju czcionki
\titlespacing{\subsection}{0ex}{0.75ex}{0.75ex} % % j/w dla tytu³ów podrozdzia³ów
\titleformat*{\subsection}{\sf\bfseries}

% Zmniejszenie odstêpów przed i za wzorami wystawionymi
\AtBeginDocument{
\addtolength{\abovedisplayskip}{-1ex}
\addtolength{\abovedisplayshortskip}{-1ex}
\addtolength{\belowdisplayskip}{-1ex}
\addtolength{\belowdisplayshortskip}{-1ex}
}
% Kilka przydatnych definicji
\newcolumntype{C}[1]{>{\centering\arraybackslash}m{#1}}
\newcommand{\razy}{\hspace{-0.5ex}\times\hspace{-0.5ex}} % mo¿e siê przydaæ


\begin{document}

\def\tablename{Tabela} % bez tej linii nazw¹ tabeli by³aby "Tablica"


\noindent
\textbf{Piotr Rowicki 320730 grupa 4 projekt 1 zadanie 35}


\section*{Wstęp} % section* oznacza rozdzia³ bez numeru (zasadne przy braku spisu treci)
Celem zadania jest przybliżanie całek podwójnych na obszarze $D$ będącym prostokątem. Aby go osiągnąć zastosujemy dwie kwadratury: złożoną kwadraturę trapezów po zmiennej $x$ opartą na $n$ podprzedziałach i złożoną kwadraturę Simosona po zmiennej $y$ opartą na $m$ podprzedziałach. Optymalna proporcja $n$ do $m$ wynosi około 40:1. Jeżeli jest spełniona, jesteśmy w stanie osiągnąć dokładniejszy wynik, nie zwiększając ilości podprzedziałów. Całki z prostych funkcji metoda przybliża w miarę szybko, bardziej wymagające  natomiast, wymagają znacznie więcej zasobów.


\section*{Wyprowadzenie metody}
Niech D=$[a,b]\times[c,d]$ będzie obszarem całkowania a $f:[a,b]\times[c,d]\xrightarrow{} R$ funkcją całkowalną, wówczas całka, której przybliżenia szukamy ma postać:
\begin{equation*} \label{eu_eqn}
\int_{c}^{d} \left( \int_{a}^{b}f(x,y)dx\right)dy.
\end{equation*}
W ogólności na wynik nie wpływa kolejność całkowania, dlatego w celu ułatwienia obliczeń całka po zmiennej x będzie całką wewnętrzną. Zastosujmy dla niej złożoną kwadraturę trapezów, traktując zmienną y jako parametr i  w ten sposób otrzymamy:
\begin{equation} \label{eu_eqn}
\int_{c}^{d} \frac{H_1}{2}\left( f(a,y) + f(b,y)+ 2\sum_{i=1}^{n-1}f(a+i H_1,y) \right)dy,
\end{equation}
gdzie $H_1=\frac{b-a}{n}$.\\
Pozostaje nam zastosować złożoną kwadraturę Simspona po zmiennej y. W wyniku tej operacji dostajemy:

\begin{multline*} \label{eu_eqn}
\frac{H_1 H_2}{6} \bigg( f(a,c) +f(a,d)+4\sum_{i=1}^{m}f(a,c+(2i-1) H_2)+2\sum_{i=1}^{m-1}f(a,c+2i H_2)+f(b,c) +f(b,d) \\
+4\sum_{i=1}^{m}f(b,c+(2i-1) H_2)+2\sum_{i=1}^{m-1}f(b,c+2i H_2)+
2\sum_{i=1}^{n-1}f(a+i H_1,c)+2\sum_{i=1}^{n-1}f(a+i  H_1,d)\\
+8\sum_{i=1}^{n-1}\sum_{j=1}^{m}f(a+i H_1,c+(2i-1)H_2)+4\sum_{i=1}^{n-1}\sum_{j=1}^{m-1}f(a+i H_1,c+2i H_2) \bigg),                                                                 
\end{multline*}
gdzie $H_2=\frac{d-c}{2m}$.
\\ Metoda przybliża szukaną całkę, wyznaczając wartość powyższej sumy.



\section*{Eksperymenty numeryczne}
Wiemy, że błąd naszej metody zależy od błędów obu zastosowanych kwadratur. Te z kolei  zależą od  ilości podprzedziałów danej kwadratury, których zwiększenie powoduje wzrost kosztu obliczeniowego. Aby zminimalizować koszty możemy ograniczyć sumaryczną liczność podprzedziałów, i rozłożyć je pomiędzy kwadratury, tak, aby błąd był jak najmniejszy.\\
W tabeli 1 przedstawione zostały błędy przybliżenia   całek podwójnych z kilku funkcji dwóch zmiennych  $f(x,y)$ na obszarach [0,1]$\times[0,1]$, dzieląc przedziały zmiennej $x$ na $n$ podprzedziałów a przedziały zmiennej $y$ na $m$ podprzedziałów. Liczby te spełniają warunek $m+n=1000$, gdzie 1000 to arbitralnie ustalony limit. Wartość $n$ na początku będzie rosła liniowo, aż do wartości 800.  Jeżeli
to nie wystarczy do wyciągnięcia wniosków, zbliżać się będzie do 1000.
\par Jak widać, zdecydowanie więcej podprzedziałów można przeznaczyć na kwadraturę mniej dokładną i uzyskać przy tym wynik bardziej dokładny. Niestety, jak pokazują przykłady  z tabeli 1, nie ma uniwersalnej proporcji $n$ do $m$ dla której błąd jest  najmniejszy. Możemy jednak przyjąć, że proporcja około 40:1 daje nam wystarczająco  zadowalające wyniki, aby stanowić w praktyce najbardziej optymalny stosunek.
\begin{table}[H]
\caption{\footnotesize ilość podprzedziałów złożonej kwadratury trapezów $n$, ilość podprzedziałów złożonej kwadratury Simpsona $m$, błąd przybliżenia całki $E(f)$ spełniający $E(f)=|I(f)-S(f)|$,  gdzie $I(f)$ - dokładna wartość całki, $S(f)$ - przybliżenie wyznaczone przez metodę z zadania}
\centering

\begin{tabular}{|C{ 20ex}|C{10ex}|C{10ex}|C{20ex}|C{10ex}|C{10ex}| } 
\hline
funkcja $f(x,y)$ & $n$ & $m$ & $E(f)$ \\

\hline
\multirow{12}{20em}{atan($xy$)}
& 200 & 800 & 3.196$\razy10^{-7}$ \\ 
& 400 & 600 & 7.991$\razy10^{-8}$\\ 
& 600 & 800 & 3.552$\razy10^{-8}$ \\ 
& 800 & 200 & 1.998$\razy10^{-8}$\\ 
& 900 & 100 & 1.578$\razy10^{-8}$ \\ 
& 950 & 50 & 1.413$\razy10^{-8}$ \\ 
& 975 & 25 & 1.293$\razy10^{-8}$ \\ 
& 985 & 15 & 9.199$\razy10^{-9}$ \\ 
& 990 & 10 & 7.125$\razy10^{-9}$ \\ 
& 993 & 7 & 7.127$\razy10^{-8}$ \\ 
\hline
\multirow{4}{20em}{3exp($x+y^2$)$xy$/7}
& 200 & 800 & 2.555$\razy10^{-5}$ \\ 
& 400 & 600 & 6.389$\razy10^{-6}$\\ 
& 600 & 800 & 2.839$\razy10^{-6}$ \\ 
& 800 & 200 & 1.597$\razy10^{-6}$\\ 
& 900 & 100 & 3.155$\razy10^{-7}$ \\ 
& 950 & 50 &  2.833$\razy10^{-7}$ \\ 
& 975 & 25 & 2.713$\razy10^{-7}$ \\ 
& 985 & 15 & 2.833$\razy10^{-7}$ \\ 
\hline
\multirow{4}{9em}{$2y^4+x^2y^2+6x^3$}
& 200 & 800 & 1.556$\razy10^{-4}$ \\ 
& 400 & 600 & 3.889$\razy10^{-5}$\\ 
& 600 & 800 & 1.728$\razy10^{-5}$ \\ 
& 800 & 200 & 9.722$\razy10^{-6}$\\ 
& 900 & 100 & 1.921$\razy10^{-6}$ \\ 
& 950 & 50 & 1.726$\razy10^{-6}$ \\ 
& 975 & 25 & 1.679$\razy10^{-6}$ \\ 
& 985 & 15 & 1.932$\razy10^{-6}$ \\ 
\hline




\end{tabular}

\end{table}
\par Ważne dla naszej metody jest to, jak szybko osiągnie zadaną poprawność. W tabeli 2 przedstawiono sumę  $n+m$ podprzedziałów obu kwadratur  dla której przybliżenie całki podwójnej z funkcji $f(x,y)$ na przedziale $[0,2]\razy[0,2]$ jest z błędem rzędu $10^{-9}$ i czas $t$ wykonania tej operacji.proporcja $n$ do $m$ wynosi 40:1.
\par Jak widać, dla prostych funkcji uzyskanie rezultatu jest w miarę szybkie, dla bardziej wymagających   natomiast, stanowi to problem.
\par 
\par 
\begin{table}[H]
    \centering
    \begin{tabular}{|C{ 30ex}|C{20ex}|C{20ex}|}
    \hline
    funkcja $f(x,y)$ & $n + m$ & $t$\\
         \hline
          atan($xy$) & 8200  &0.0.019874\\
         \hline
        
        3exp($x+y^2$)$xy$/7 & 262400 &19.621320\\
         \hline
        $2y^4+x^2y^2+6x^3$& 98400  &8.732748\\
         \hline
        sqrt($x^2+y^2)$ & 12300  &0.030007\\
         \hline
        2$y$sin($x$)+ $x$cos($2y$)& 16400  &0.069968\\
         \hline
         $x^2$/($y^2$+1) & 16400  &0.034005\\
         \hline
         
    \end{tabular}
    \caption{\footnotesize suma ilości podprzedziałów obu kwadratur $m+n$ dla której otrzymaliśmy szukaną poprawność, czas wykonoywania obliczeń $t$ w sekundach }
    \label{tab:my_label}
\end{table}

\end{document}
